% @ telmo - original text was reworked!
% BRIEF EXPLANATION %
This section is reserved to enumerate and describe the project objectives.
Those objectives are related to the \textit{feasibility stage}, this paper's current iteration, and the \textit{implementation stage}, regarding the project's final milestones.

% FEASIBILITY OBJECTIVES % 
\noindent \textbf{Feasibility Objectives:}
\begin{itemize}
    \item Environment - achieve an environment, through tools and components described in following sections, where the development of software, its execution, and its benchmarking can be preformed.
    \item Feasibility - pertains to the current iteration's main goal, via a feasibility proof validating both the capability of the ecosystem to provide results and laying the ground for the \textit{implementation stage}.
    \item Reflections - discussion on the first findings and major milestones.
\end{itemize}

% IMPLEMENTATION OBJECTIVES %
\noindent \textbf{Implementation Objectives:}
\begin{itemize}
    \item Use Cases - development and analysis of \textit{XDP} programs, tackling concrete scenarios  regarding packet processing and filtering such as, filtering packets by their IPv4/IPv6 header fields, packet transmission on specific ports and dropping packets from exploitable protocols.
    \item Primitive Actions - development and analysis of \textit{XDP} primitive actions, this constitutes the main goal on the \textit{implementation stage}.
    \item Benchmarks - measure the \#packets/time ratio for each of the primitive actions.
    \item Results - analysis on how the results obtained relate to the theoretical limits between the different \textit{XDP} modes. 
\end{itemize}

% SOME CONSIDERATIONS %
These goals are explored in greater detail on the following sections.

% @ telmo - original text by henrique:
% At the end of the introduction, we address the project objectives in a summarized manner. In this section, we will address them again, but in a more complete way:

% 1 - Evaluation of Performance Enhancements: The primary goal is to evaluate the performance enhancements offered by XDP within the Linux ecosystem. Through comprehensive testing and analysis, we aim to know all the improvements XDP can offer in several situations, such as packet processing speed, latency reduction and throughput optimization achieved.

% 2 - Custom Packet Filtering Implementation: Another important objective is to implement custom packet filtering functionalities using XDP. By developing tailored packet filtering logic, we intend to demonstrate the flexibility and versatility of XDP in addressing specific network security and traffic management requirements, which involves designing and deploying XDP programs to filter, forward or drop packets based on user-defined criteria in an efficient way.

% 3 - Exploration of Use Cases: The project aims to explore various use cases where XDP can provide tangible benefits. This includes scenarios such as DDoS mitigation, intrusion detection and prevention, load balancing, and network function virtualization (NFV). By deploying XDP-based solutions in diverse networking environments, we seek to identify and document the specific scenarios where XDP excels.

% 4 - Performance Benchmarking: An essential aspect of the project involves conducting performance benchmarking of XDP-based solutions. This includes comparing the performance metrics of XDP-enabled applications against traditional networking approaches. Through systematic testing under different workload conditions, we aim to provide empirical evidence of XDP's superiority in terms of speed, scalability, and resource efficiency.

% 5 - Real-world Deployment Assessment: In addition to performance evaluation in controlled environments, the project aims to assess the suitability of XDP for real-world deployment scenarios. This involves testing XDP-based solutions in production-like environments and evaluating factors such as ease of integration, compatibility with existing infrastructure, and overall operational efficiency.

% (...)? achieving these objectives, the project aims to provide valuable insights into the capabilities of XDP and its potential to revolutionize packet processing in Linux-based networking environments. Through empirical analysis and practical experimentation, we seek to contribute to the advancement of networking technologies and facilitate the adoption of efficient packet processing solutions in the industry.