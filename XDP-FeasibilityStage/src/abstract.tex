\begin{abstract}
% PROJECT STATEMENT BY PROF. PRIOR %
The project aims to demonstrate the use of \textit{XDP} in Linux to accelerate packet processing and filtering while analyzing its performance.
% PROJECT EXTERNAL STRUCTURE %
The project is divided in two parts, the \textit{feasibility stage} and the \textit{implementation stage}.
The current iteration pertains to the first stage where the objectives, environment and the initial findings are stated.
% THE TECHNOLOGY %
\textit{eXpress Data Path (XDP)} is present in the Linux ecosystem since its 4.8 version. 
A technology that leverages \textit{extended Berkeley Packet Filter (eBPF)} to execute a finer and faster control on packet processing. 
Traditional packet processing methods struggle to mitigate latency and offer suboptimal performance, \textit{XDP} addresses these problems through processing in earlier layers on the network stack.
\end{abstract}
%
\begin{IEEEkeywords}
eXpress Data Path, XDP, linux networking, packet processing, eBPF, extended Berkeley Packet Filter, cybersecurity, benchmarks
\end{IEEEkeywords}