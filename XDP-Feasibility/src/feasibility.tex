% PLEASE READ THIS 
% WE DID NOT YET TEST THIS PROPOSAL

% FT GOALS %
This section discusses the methodology and conclusions derived from the \textit{Feasibility Test}, from this point on referred as \textit{FT}.
The \textit{FT} aims to validate \textbf{(1)} the successful setup of the environment, \textbf{(2)} render the possibility to develop \textit{XDP} programs, and \textbf{(3)} visualize \textit{eBPF} objects running.

% FT METHODOLOGY %
% insert better definitions for the modes %
\textit{XDP} can be implemented through three different modes:
\begin{itemize}
    \item \textbf{Generic XDP:}   XDP programs are loaded into the kernel;
    \item \textbf{Native XDP:}    XDP program is loaded by the network card driver;
    \item \textbf{Offloaded XDP:} XDP program is loaded directly on the NIC bypassing the CPU.
\end{itemize}

For the \textit{FT}, we focused on the \textbf{Generic XDP} as it can accomplish the mentioned goals without the need for supporting hardware.
The \textit{FT} was conducted in the environment described in the \textit{Tools \& Components} section.
It models the undemanding scenario where all IPv4 are dropped and all IPv6 are passed.

The steps followed were:
\begin{itemize}
    \item Write the \textit{XDP} program.
    \item Build and dump the \textit{eBPF} object.
    \item Load the \textit{eBPF} object.
    \item Experiment on the running object.
\end{itemize}
% CODE LISTING %
The source code for the \textit{XDP} program is as follows:
\begin{lstlisting}[language=C]
#include <linux/bpf.h>
#include <linux/if_ether.h>
#include <linux/ip.h>
#include <linux/ipv6.h>

SEC("xdp_prog")
int xdp_pass_ipv6(struct xdp_md* ctx) {
    void* data_end = (void*)(long)ctx->data_end;
    void* data     = (void*)(long)ctx->data;
    if (data + sizeof(struct ethhdr) > data_end) { 
        return XDP_DROP;
    }
    struct ethhdr* eth = data;
    if (eth->h_proto == htons(ETH_P_IPV6)) {
        return XDP_PASS;
    }
    return XDP_DROP;
}

char _license[] SEC("license") = "GPL";
\end{lstlisting}

% CODE DESCRIPTION %
\textbf{SEC} enables the placement of compiled object fragments into different ELF sections.\
The function \textbf{xdp\_pass\_ipv6()} accepts a parameter of type \textbf{struct xdp\_md*}.
This struct allows the initialization of pointers that reference crucial header delimiters, asserting the access of legal fields within the header.
If the protocol used is IPv6, the packet should be passed, essentially returning \textbf{XDP\_PASS}.
Otherwise, the packets are dropped, returning \textbf{XDP\_DROP}.
Finally, the last line regards to the license associated with the program, which in this case, is GPL (General Public License).

% FT RESULT %
Aided by \textbf{network debugging tools}, we preformed a plethora of tests on the interface previously loaded with the \textit{eBPF} object, all corroborating that the desired property was modeled successfully and, therefore, deriving the conclusion that the \textit{FT} goals were achieved.